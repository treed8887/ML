% Options for packages loaded elsewhere
\PassOptionsToPackage{unicode}{hyperref}
\PassOptionsToPackage{hyphens}{url}
%
\documentclass[
]{article}
\usepackage{amsmath,amssymb}
\usepackage{lmodern}
\usepackage{ifxetex,ifluatex}
\ifnum 0\ifxetex 1\fi\ifluatex 1\fi=0 % if pdftex
  \usepackage[T1]{fontenc}
  \usepackage[utf8]{inputenc}
  \usepackage{textcomp} % provide euro and other symbols
\else % if luatex or xetex
  \usepackage{unicode-math}
  \defaultfontfeatures{Scale=MatchLowercase}
  \defaultfontfeatures[\rmfamily]{Ligatures=TeX,Scale=1}
\fi
% Use upquote if available, for straight quotes in verbatim environments
\IfFileExists{upquote.sty}{\usepackage{upquote}}{}
\IfFileExists{microtype.sty}{% use microtype if available
  \usepackage[]{microtype}
  \UseMicrotypeSet[protrusion]{basicmath} % disable protrusion for tt fonts
}{}
\makeatletter
\@ifundefined{KOMAClassName}{% if non-KOMA class
  \IfFileExists{parskip.sty}{%
    \usepackage{parskip}
  }{% else
    \setlength{\parindent}{0pt}
    \setlength{\parskip}{6pt plus 2pt minus 1pt}}
}{% if KOMA class
  \KOMAoptions{parskip=half}}
\makeatother
\usepackage{xcolor}
\IfFileExists{xurl.sty}{\usepackage{xurl}}{} % add URL line breaks if available
\IfFileExists{bookmark.sty}{\usepackage{bookmark}}{\usepackage{hyperref}}
\hypersetup{
  pdftitle={ROS: Chapter4 Statistical Inference: Part1},
  pdfauthor={Prof.~Kapitula},
  hidelinks,
  pdfcreator={LaTeX via pandoc}}
\urlstyle{same} % disable monospaced font for URLs
\usepackage[margin=1in]{geometry}
\usepackage{color}
\usepackage{fancyvrb}
\newcommand{\VerbBar}{|}
\newcommand{\VERB}{\Verb[commandchars=\\\{\}]}
\DefineVerbatimEnvironment{Highlighting}{Verbatim}{commandchars=\\\{\}}
% Add ',fontsize=\small' for more characters per line
\usepackage{framed}
\definecolor{shadecolor}{RGB}{248,248,248}
\newenvironment{Shaded}{\begin{snugshade}}{\end{snugshade}}
\newcommand{\AlertTok}[1]{\textcolor[rgb]{0.94,0.16,0.16}{#1}}
\newcommand{\AnnotationTok}[1]{\textcolor[rgb]{0.56,0.35,0.01}{\textbf{\textit{#1}}}}
\newcommand{\AttributeTok}[1]{\textcolor[rgb]{0.77,0.63,0.00}{#1}}
\newcommand{\BaseNTok}[1]{\textcolor[rgb]{0.00,0.00,0.81}{#1}}
\newcommand{\BuiltInTok}[1]{#1}
\newcommand{\CharTok}[1]{\textcolor[rgb]{0.31,0.60,0.02}{#1}}
\newcommand{\CommentTok}[1]{\textcolor[rgb]{0.56,0.35,0.01}{\textit{#1}}}
\newcommand{\CommentVarTok}[1]{\textcolor[rgb]{0.56,0.35,0.01}{\textbf{\textit{#1}}}}
\newcommand{\ConstantTok}[1]{\textcolor[rgb]{0.00,0.00,0.00}{#1}}
\newcommand{\ControlFlowTok}[1]{\textcolor[rgb]{0.13,0.29,0.53}{\textbf{#1}}}
\newcommand{\DataTypeTok}[1]{\textcolor[rgb]{0.13,0.29,0.53}{#1}}
\newcommand{\DecValTok}[1]{\textcolor[rgb]{0.00,0.00,0.81}{#1}}
\newcommand{\DocumentationTok}[1]{\textcolor[rgb]{0.56,0.35,0.01}{\textbf{\textit{#1}}}}
\newcommand{\ErrorTok}[1]{\textcolor[rgb]{0.64,0.00,0.00}{\textbf{#1}}}
\newcommand{\ExtensionTok}[1]{#1}
\newcommand{\FloatTok}[1]{\textcolor[rgb]{0.00,0.00,0.81}{#1}}
\newcommand{\FunctionTok}[1]{\textcolor[rgb]{0.00,0.00,0.00}{#1}}
\newcommand{\ImportTok}[1]{#1}
\newcommand{\InformationTok}[1]{\textcolor[rgb]{0.56,0.35,0.01}{\textbf{\textit{#1}}}}
\newcommand{\KeywordTok}[1]{\textcolor[rgb]{0.13,0.29,0.53}{\textbf{#1}}}
\newcommand{\NormalTok}[1]{#1}
\newcommand{\OperatorTok}[1]{\textcolor[rgb]{0.81,0.36,0.00}{\textbf{#1}}}
\newcommand{\OtherTok}[1]{\textcolor[rgb]{0.56,0.35,0.01}{#1}}
\newcommand{\PreprocessorTok}[1]{\textcolor[rgb]{0.56,0.35,0.01}{\textit{#1}}}
\newcommand{\RegionMarkerTok}[1]{#1}
\newcommand{\SpecialCharTok}[1]{\textcolor[rgb]{0.00,0.00,0.00}{#1}}
\newcommand{\SpecialStringTok}[1]{\textcolor[rgb]{0.31,0.60,0.02}{#1}}
\newcommand{\StringTok}[1]{\textcolor[rgb]{0.31,0.60,0.02}{#1}}
\newcommand{\VariableTok}[1]{\textcolor[rgb]{0.00,0.00,0.00}{#1}}
\newcommand{\VerbatimStringTok}[1]{\textcolor[rgb]{0.31,0.60,0.02}{#1}}
\newcommand{\WarningTok}[1]{\textcolor[rgb]{0.56,0.35,0.01}{\textbf{\textit{#1}}}}
\usepackage{graphicx}
\makeatletter
\def\maxwidth{\ifdim\Gin@nat@width>\linewidth\linewidth\else\Gin@nat@width\fi}
\def\maxheight{\ifdim\Gin@nat@height>\textheight\textheight\else\Gin@nat@height\fi}
\makeatother
% Scale images if necessary, so that they will not overflow the page
% margins by default, and it is still possible to overwrite the defaults
% using explicit options in \includegraphics[width, height, ...]{}
\setkeys{Gin}{width=\maxwidth,height=\maxheight,keepaspectratio}
% Set default figure placement to htbp
\makeatletter
\def\fps@figure{htbp}
\makeatother
\setlength{\emergencystretch}{3em} % prevent overfull lines
\providecommand{\tightlist}{%
  \setlength{\itemsep}{0pt}\setlength{\parskip}{0pt}}
\setcounter{secnumdepth}{-\maxdimen} % remove section numbering
\ifluatex
  \usepackage{selnolig}  % disable illegal ligatures
\fi

\title{ROS: Chapter4 Statistical Inference: Part1}
\author{Prof.~Kapitula}
\date{9/23/2021}

\begin{document}
\maketitle

\hypertarget{needed-packages}{%
\subsubsection*{Needed packages}\label{needed-packages}}
\addcontentsline{toc}{subsubsection}{Needed packages}

\begin{Shaded}
\begin{Highlighting}[]
\FunctionTok{library}\NormalTok{(tidyverse)}
\FunctionTok{library}\NormalTok{(infer)}
\FunctionTok{library}\NormalTok{(janitor)}
\FunctionTok{library}\NormalTok{(matrixStats)}
\end{Highlighting}
\end{Shaded}

Working through examples in \emph{Regression and Other Stories} by
Gelman, Hill and Vehtari. Original RMD files were Solomon Kurz's
versions of examples from ROS that he originally edited but they have
been heavily edited by Professor Kapitula.

\hypertarget{sampling-distributions-and-generative-models}{%
\subsection{Sampling distributions and generative
models}\label{sampling-distributions-and-generative-models}}

\hypertarget{sampling-measurement-error-and-model-error}{%
\subsubsection{Sampling, measurement error, and model
error}\label{sampling-measurement-error-and-model-error}}

Standard paradigms for thinking about the role of inference:

\begin{itemize}
\item
  In the sampling model, we are interested in learning some
  characteristics of a population (for example, the mean and standard
  deviation of the heights of all women in the United States), which we
  must estimate from a sample, or subset, of that population.
\item
  In the measurement error model, we are interested in learning aspects
  of some underlying pattern or law, simple example:
\end{itemize}

\[y_i=a+bX_i+\epsilon_i\] So the \(\epsilon_i\) is error.

\begin{itemize}
\tightlist
\item
  Model error refers to the inevitable imperfections of the models that
  we apply to real data.
\end{itemize}

In practice, we often consider all three issues when constructing and
working with a statistical model. We will consider the standard set up
for regression where the data can be thought of as a sample from some
population or distribution. The \(\epsilon_i\) will be thought of
typically as independent, identically distributed with some some
distribution, and zero mean. For example,
\[\epsilon_i \sim N(0,\sigma^2) \]

\hypertarget{the-mean-only-model}{%
\subsection{The mean only model}\label{the-mean-only-model}}

Suppose we have

\[y_i=\mu+\epsilon_i\] This is sometimes called the null model, or the
grand mean only model.

\hypertarget{the-sampling-distribution.}{%
\subsubsection{The sampling
distribution.}\label{the-sampling-distribution.}}

\begin{quote}
The \emph{sampling distribution} is the set of possible datasets that
could have been observed if the data collection process had been
re-done, along with the probabilities of these possible values\ldots.

the sampling distribution in general will not typically be known, as it
depends on aspects of the population, not merely on the observed data
(p.~50, \emph{emphasis} in the original).
\end{quote}

\hypertarget{estimates-standard-errors-and-confidence-intervals}{%
\subsection{4.2 Estimates, standard errors, and confidence
intervals}\label{estimates-standard-errors-and-confidence-intervals}}

\hypertarget{parameters-estimands-and-estimates.}{%
\subsubsection{4.2.1 Parameters, estimands, and
estimates.}\label{parameters-estimands-and-estimates.}}

\begin{quote}
In statistics jargon, \emph{parameters} are the unknown numbers that
determine a statistical model. For example, consider the model
\(yi = a + b x_i + \epsilon_i\), in which the errors \(\epsilon_i\) are
normally distributed with mean 0 and standard deviation \(\sigma\). The
parameters in this model are \(a\), \(b\), and \(\sigma\). The
parameters \(a\) and \(b\) are called \emph{coefficients}, and
\(\sigma\) is a called a \emph{scale} or \emph{variance
parameter}\ldots.

An \emph{estimand}, or \emph{quantity of interest}, is some summary of
parameters or data that somebody is interested in estimating. For
example, in the regression model, \(yi = a + b x_i + \text{error}\), the
parameters \(a\) and \(b\) might be of interest--\(a\) is the intercept
of the model, the predicted value of \(y\) when \(x = 0\); and \(b\) is
the slope, the predicted difference in \(y\), comparing two data points
that differ by 1 in \(x\). Other quantities of interest could be
predicted outcomes for particular new data points, or combinations of
predicted values such as sums, differences, averages, and ratios.

We use the data to construct estimates of parameters and other
quantities of interest. (pp.~50--51, \emph{emphasis} in the original)
\end{quote}

\hypertarget{standard-errors-inferential-uncertainty-and-confidence-intervals.}{%
\subsubsection{4.2.2 Standard errors, inferential uncertainty, and
confidence
intervals.}\label{standard-errors-inferential-uncertainty-and-confidence-intervals.}}

The standard error is a measure of the variation in an estimate and gets
smaller as sample size gets larger, converging on zero as the sample
increases in size. \textgreater{} \textgreater{} The \emph{confidence
interval} represents a range of values of a parameter or quantity of
interest that are roughly consistent with the data, given the assumed
sampling distribution. If the model is correct, then in repeated
applications the 50\% and 95\% confidence intervals will include the
true value 50\% and 95\% of the time. (p.~51, \emph{emphasis} in the
original)

\[ statistic + or - multiplier*SE(statistic)\]

You can find the authors' simulation code for Figure 4.2 in the
\texttt{coverage.Rmd} file within the \texttt{Coverage} folder. Here we
adjust it a bit for a \textbf{tidyverse}-style work flow.

\begin{Shaded}
\begin{Highlighting}[]
\CommentTok{\# how many simulations would you like?}
\NormalTok{nsims }\OtherTok{\textless{}{-}} \DecValTok{100}
\NormalTok{n }\OtherTok{\textless{}{-}}\DecValTok{10}
\CommentTok{\# set the true data{-}generating parameters}
\NormalTok{mu }\OtherTok{\textless{}{-}} \DecValTok{6}
\NormalTok{sigma }\OtherTok{\textless{}{-}} \DecValTok{4}
\FunctionTok{set.seed}\NormalTok{(}\DecValTok{410}\NormalTok{)}

\CommentTok{\# simulate}
\NormalTok{sims }\OtherTok{\textless{}{-}} \FunctionTok{matrix}\NormalTok{(}\FunctionTok{rnorm}\NormalTok{(n}\SpecialCharTok{*}\NormalTok{nsims, }\AttributeTok{mean =}\NormalTok{ mu, }\AttributeTok{sd =}\NormalTok{ sigma),nsims,n)}
\CommentTok{\# sims}
\NormalTok{ybar }\OtherTok{\textless{}{-}} \FunctionTok{rowMeans}\NormalTok{(sims)}
\NormalTok{s }\OtherTok{\textless{}{-}} \FunctionTok{rowSds}\NormalTok{(sims)}
\end{Highlighting}
\end{Shaded}

\begin{Shaded}
\begin{Highlighting}[]
\NormalTok{d }\OtherTok{\textless{}{-}}
  \FunctionTok{tibble}\NormalTok{(}\AttributeTok{i =} \DecValTok{1}\SpecialCharTok{:}\NormalTok{nsims,ybar,s) }\SpecialCharTok{\%\textgreater{}\%} 
  \FunctionTok{mutate}\NormalTok{(}\AttributeTok{ll95 =}\NormalTok{ ybar }\SpecialCharTok{{-}} \DecValTok{2} \SpecialCharTok{*}\NormalTok{ s}\SpecialCharTok{/}\FunctionTok{sqrt}\NormalTok{(n),}
         \AttributeTok{ll50 =}\NormalTok{ ybar }\SpecialCharTok{{-}} \FloatTok{0.67} \SpecialCharTok{*}\NormalTok{ s}\SpecialCharTok{/}\FunctionTok{sqrt}\NormalTok{(n),}
         \AttributeTok{ul50 =}\NormalTok{ ybar }\SpecialCharTok{+} \FloatTok{0.67} \SpecialCharTok{*}\NormalTok{ s}\SpecialCharTok{/}\FunctionTok{sqrt}\NormalTok{(n),}
         \AttributeTok{ul95 =}\NormalTok{ ybar }\SpecialCharTok{+} \DecValTok{2} \SpecialCharTok{*}\NormalTok{ s}\SpecialCharTok{/}\FunctionTok{sqrt}\NormalTok{(n)) }

\CommentTok{\# plot}
\NormalTok{d }\SpecialCharTok{\%\textgreater{}\%} 
  \FunctionTok{ggplot}\NormalTok{(}\FunctionTok{aes}\NormalTok{(}\AttributeTok{x =}\NormalTok{ i, }\AttributeTok{y =}\NormalTok{ ybar)) }\SpecialCharTok{+}
  \FunctionTok{geom\_hline}\NormalTok{(}\AttributeTok{yintercept =}\NormalTok{ mu, }\AttributeTok{color =} \StringTok{"grey75"}\NormalTok{, }\AttributeTok{size =} \DecValTok{1}\SpecialCharTok{/}\DecValTok{4}\NormalTok{) }\SpecialCharTok{+}
  \FunctionTok{geom\_pointrange}\NormalTok{(}\FunctionTok{aes}\NormalTok{(}\AttributeTok{ymin =}\NormalTok{ ll95, }\AttributeTok{ymax =}\NormalTok{ ul95),}
                  \AttributeTok{size =} \DecValTok{1}\SpecialCharTok{/}\DecValTok{4}\NormalTok{, }\AttributeTok{fatten =} \DecValTok{2}\SpecialCharTok{/}\DecValTok{3}\NormalTok{) }\SpecialCharTok{+}
  \FunctionTok{geom\_linerange}\NormalTok{(}\FunctionTok{aes}\NormalTok{(}\AttributeTok{ymin =}\NormalTok{ ll50, }\AttributeTok{ymax =}\NormalTok{ ul50),}
                 \AttributeTok{size =} \DecValTok{1}\SpecialCharTok{/}\DecValTok{2}\NormalTok{) }\SpecialCharTok{+}
  \FunctionTok{labs}\NormalTok{(}\AttributeTok{title =} \StringTok{"Simulation of coverage of confidence intervals"}\NormalTok{,}
       \AttributeTok{subtitle =} \FunctionTok{paste}\NormalTok{(}\StringTok{"The horizontal line shows the true parameter value, and dots and vertical lines show}\SpecialCharTok{\textbackslash{}n}\StringTok{estimates and confidence intervals obtained from 100 random simulations from the}\SpecialCharTok{\textbackslash{}n}\StringTok{sampling distribution. (n="}\NormalTok{,n,}\StringTok{")"}\NormalTok{),}
       \AttributeTok{x =} \StringTok{"Simulation index"}\NormalTok{,}
       \AttributeTok{y =} \StringTok{"Estimate, 50\%, and 95\%}\SpecialCharTok{\textbackslash{}n}\StringTok{confidence interval"}\NormalTok{)}
\end{Highlighting}
\end{Shaded}

\begin{center}\includegraphics[width=468px]{ROS_chapter_4_single_quant_part1_files/figure-latex/unnamed-chunk-4-1} \end{center}

To check, here's the percentage of 95\% intervals containing the
data-generating population mean, \texttt{mu}.

\begin{Shaded}
\begin{Highlighting}[]
\CommentTok{\# when we sum a logical, we count}
\NormalTok{d }\SpecialCharTok{\%\textgreater{}\%} 
  \FunctionTok{summarise}\NormalTok{(}\AttributeTok{percent =} \FunctionTok{sum}\NormalTok{(ll95 }\SpecialCharTok{\textless{}=}\NormalTok{ mu }\SpecialCharTok{\&}\NormalTok{ mu }\SpecialCharTok{\textless{}=}\NormalTok{ ul95 )}\SpecialCharTok{/}\NormalTok{nsims)}
\end{Highlighting}
\end{Shaded}

\begin{verbatim}
## # A tibble: 1 x 1
##   percent
##     <dbl>
## 1    0.93
\end{verbatim}

And here's the percentage of 50\% intervals containing the
data-generating population mean, \texttt{mu}.

\begin{Shaded}
\begin{Highlighting}[]
\NormalTok{d }\SpecialCharTok{\%\textgreater{}\%} 
  \FunctionTok{summarise}\NormalTok{(}\AttributeTok{percent =} \FunctionTok{sum}\NormalTok{(ll50 }\SpecialCharTok{\textless{}=}\NormalTok{ mu }\SpecialCharTok{\&}\NormalTok{ ul50 }\SpecialCharTok{\textgreater{}=}\NormalTok{ mu)}\SpecialCharTok{/}\NormalTok{nsims)}
\end{Highlighting}
\end{Shaded}

\begin{verbatim}
## # A tibble: 1 x 1
##   percent
##     <dbl>
## 1    0.45
\end{verbatim}

Both showed good coverage.

The confidence level is the success rate of the method for calculating
the confidence interval.

\hypertarget{standard-errors-and-confidence-intervals-for-averages-and-proportions.}{%
\subsubsection{4.2.3 Standard errors and confidence intervals for
averages and
proportions.}\label{standard-errors-and-confidence-intervals-for-averages-and-proportions.}}

\begin{quote}
When estimating the mean of an infinite population, given a simple
random sample of size \(n\), the standard error is \(\sigma / \sqrt n\),
where \(\sigma\) is the standard deviation of the measurements in the
population. This property holds regardless of any assumption about the
shape of the sampling distribution, but the standard error might be less
informative for sampling distributions that are far from normal.

A proportion is a special case of an average in which the data are 0's
and 1's. Consider a survey of size \(n\) with \(y\) Yes responses and
\(n - y\) No responses. The estimated proportion of the population who
would answer Yes to this survey is \(\hat p = y / n\), and the standard
error of this estimate is \(\sqrt{\hat p (1 - \hat p) / n}\). If \(p\)
is near 0.5, we can approximate this by \(0.5 / \sqrt n\). (pp.~51--52)
\end{quote}

Consider a case where out of a random sample of 1,000, 700 opposed the
death penalty and 300 supported it, we can use the formula
\(\sqrt{\hat p (1 - \hat p) / n}\) to compute the 95\% confidence
intervals in \textbf{R} like so.

Below is a confidence interval comuted using a normal approximation.

\begin{Shaded}
\begin{Highlighting}[]
\NormalTok{n }\OtherTok{\textless{}{-}} \DecValTok{1000}
\NormalTok{y }\OtherTok{\textless{}{-}} \DecValTok{700}

\NormalTok{estimate }\OtherTok{\textless{}{-}}\NormalTok{ y}\SpecialCharTok{/}\NormalTok{n}

\NormalTok{se }\OtherTok{\textless{}{-}} \FunctionTok{sqrt}\NormalTok{(estimate }\SpecialCharTok{*}\NormalTok{ (}\DecValTok{1} \SpecialCharTok{{-}}\NormalTok{ estimate) }\SpecialCharTok{/}\NormalTok{ n)}

\NormalTok{estimate }\SpecialCharTok{+} \FunctionTok{qnorm}\NormalTok{(}\FunctionTok{c}\NormalTok{(.}\DecValTok{025}\NormalTok{, .}\DecValTok{975}\NormalTok{), }\AttributeTok{mean =} \DecValTok{0}\NormalTok{, }\AttributeTok{sd =} \DecValTok{1}\NormalTok{) }\SpecialCharTok{*}\NormalTok{ se}
\end{Highlighting}
\end{Shaded}

\begin{verbatim}
## [1] 0.6715974 0.7284026
\end{verbatim}

\hypertarget{standard-error-and-confidence-interval-for-a-proportion-when-y-0-or-y-n.}{%
\subsubsection{\texorpdfstring{4.2.4 Standard error and confidence
interval for a proportion when \(y = 0\) or
\(y = n\).}{4.2.4 Standard error and confidence interval for a proportion when y = 0 or y = n.}}\label{standard-error-and-confidence-interval-for-a-proportion-when-y-0-or-y-n.}}

As a proportion approaches zero or one, the method used above tends to
break down. ``A standard and reasonable quick correction for
constructing a 95\% interval when \(y\) or \(n - y\) is near zero is to
use the estimate \(\hat p = \frac{y + 2}{n + 4}\) with standard error
\(\sqrt{\hat p(1 − \hat p)/(n + 4)}\)'' (p.~52). Here's how this would
work when \(y = 0\) and \(n = 75\).

\begin{Shaded}
\begin{Highlighting}[]
\NormalTok{n }\OtherTok{\textless{}{-}} \DecValTok{75}
\NormalTok{y }\OtherTok{\textless{}{-}} \DecValTok{0}

\CommentTok{\# probability, when we use the +4 method we add two successes}
\CommentTok{\# and two failures}
\NormalTok{(estimate }\OtherTok{\textless{}{-}}\NormalTok{ (y }\SpecialCharTok{+} \DecValTok{2}\NormalTok{) }\SpecialCharTok{/}\NormalTok{ (n }\SpecialCharTok{+} \DecValTok{4}\NormalTok{))}
\end{Highlighting}
\end{Shaded}

\begin{verbatim}
## [1] 0.02531646
\end{verbatim}

\begin{Shaded}
\begin{Highlighting}[]
\CommentTok{\# se}
\NormalTok{(}\FunctionTok{sqrt}\NormalTok{(estimate }\SpecialCharTok{*}\NormalTok{ (}\DecValTok{1} \SpecialCharTok{{-}}\NormalTok{ estimate) }\SpecialCharTok{/}\NormalTok{ (n }\SpecialCharTok{+} \DecValTok{4}\NormalTok{)))}
\end{Highlighting}
\end{Shaded}

\begin{verbatim}
## [1] 0.01767338
\end{verbatim}

\begin{Shaded}
\begin{Highlighting}[]
\CommentTok{\# 95\% CI}
\NormalTok{estimate }\SpecialCharTok{+} \FunctionTok{qnorm}\NormalTok{(}\FunctionTok{c}\NormalTok{(.}\DecValTok{025}\NormalTok{, .}\DecValTok{975}\NormalTok{), }\AttributeTok{mean =} \DecValTok{0}\NormalTok{, }\AttributeTok{sd =} \DecValTok{1}\NormalTok{) }\SpecialCharTok{*}\NormalTok{ se}
\end{Highlighting}
\end{Shaded}

\begin{verbatim}
## [1] -0.003086121  0.053719032
\end{verbatim}

However, since ``it makes no sense for the interval for a proportion to
contain negative values, so we truncate the interval to obtain {[}0,
0.054{]}{[}\^{}1{]}'' (p.~52).

Another, method to use to get the CI is to use prop.test. prop.test uses
the Wilson Interval, which will work fairly well with small y or n-y.
\url{https://en.wikipedia.org/wiki/Binomial_proportion_confidence_interval\#Wilson_score_interval_with_continuity_correction}

\begin{Shaded}
\begin{Highlighting}[]
\NormalTok{n }\OtherTok{\textless{}{-}} \DecValTok{75}
\NormalTok{y }\OtherTok{\textless{}{-}} \DecValTok{0}

\FunctionTok{prop.test}\NormalTok{(y,n)}
\end{Highlighting}
\end{Shaded}

\begin{verbatim}
## 
##  1-sample proportions test with continuity correction
## 
## data:  y out of n, null probability 0.5
## X-squared = 73.013, df = 1, p-value < 2.2e-16
## alternative hypothesis: true p is not equal to 0.5
## 95 percent confidence interval:
##  0.00000000 0.06071113
## sample estimates:
## p 
## 0
\end{verbatim}

\hypertarget{standard-error-for-a-comparison.}{%
\subsubsection{4.2.5 Standard error for a
comparison.}\label{standard-error-for-a-comparison.}}

The formula to compute the standard error of the difference of two
independent quantities follows the form

\[\text{standard error of the difference} = \sqrt{\text{se}_1^2 + \text{se}_2^2}.\]

Consider a survey of 400 men, 57\% of whom said they'd vote Republican,
and 600 women, 45\% of whom said they'd vote Republican.

\begin{Shaded}
\begin{Highlighting}[]
\CommentTok{\# men}
\NormalTok{n }\OtherTok{\textless{}{-}} \DecValTok{400}
\NormalTok{y }\OtherTok{\textless{}{-}}\NormalTok{ n }\SpecialCharTok{*}\NormalTok{ .}\DecValTok{57}

\NormalTok{estimate\_men }\OtherTok{\textless{}{-}}\NormalTok{ y}\SpecialCharTok{/}\NormalTok{n}

\NormalTok{se\_men }\OtherTok{\textless{}{-}} \FunctionTok{sqrt}\NormalTok{(estimate\_men }\SpecialCharTok{*}\NormalTok{ (}\DecValTok{1} \SpecialCharTok{{-}}\NormalTok{ estimate\_men) }\SpecialCharTok{/}\NormalTok{ n)}

\CommentTok{\# women}
\NormalTok{n }\OtherTok{\textless{}{-}} \DecValTok{600}
\NormalTok{y }\OtherTok{\textless{}{-}}\NormalTok{ n }\SpecialCharTok{*}\NormalTok{ .}\DecValTok{45}

\NormalTok{estimate\_women }\OtherTok{\textless{}{-}}\NormalTok{ y}\SpecialCharTok{/}\NormalTok{n}

\NormalTok{se\_women }\OtherTok{\textless{}{-}} \FunctionTok{sqrt}\NormalTok{(estimate\_women }\SpecialCharTok{*}\NormalTok{ (}\DecValTok{1} \SpecialCharTok{{-}}\NormalTok{ estimate\_women) }\SpecialCharTok{/}\NormalTok{ n)}

\CommentTok{\# estimated gender gap}
\NormalTok{estimate\_men }\SpecialCharTok{{-}}\NormalTok{ estimate\_women}
\end{Highlighting}
\end{Shaded}

\begin{verbatim}
## [1] 0.12
\end{verbatim}

\begin{Shaded}
\begin{Highlighting}[]
\CommentTok{\# se difference}
\FunctionTok{sqrt}\NormalTok{(se\_men}\SpecialCharTok{\^{}}\DecValTok{2} \SpecialCharTok{+}\NormalTok{ se\_women}\SpecialCharTok{\^{}}\DecValTok{2}\NormalTok{)}
\end{Highlighting}
\end{Shaded}

\begin{verbatim}
## [1] 0.03201953
\end{verbatim}

\begin{Shaded}
\begin{Highlighting}[]
\CommentTok{\# can use prop.test to get CI for difference}
\FunctionTok{prop.test}\NormalTok{(}\FunctionTok{matrix}\NormalTok{(}\FunctionTok{c}\NormalTok{(.}\DecValTok{57}\SpecialCharTok{*}\DecValTok{400}\NormalTok{,.}\DecValTok{45}\SpecialCharTok{*}\DecValTok{600}\NormalTok{,}\DecValTok{400}\FloatTok{{-}.57}\SpecialCharTok{*}\DecValTok{400}\NormalTok{,}\DecValTok{600}\FloatTok{{-}.45}\SpecialCharTok{*}\DecValTok{600}\NormalTok{),}\DecValTok{2}\NormalTok{,}\DecValTok{2}\NormalTok{))}
\end{Highlighting}
\end{Shaded}

\begin{verbatim}
## 
##  2-sample test for equality of proportions with continuity correction
## 
## data:  matrix(c(0.57 * 400, 0.45 * 600, 400 - 0.57 * 400, 600 - 0.45 * 600), 2, 2)
## X-squared = 13.348, df = 1, p-value = 0.0002586
## alternative hypothesis: two.sided
## 95 percent confidence interval:
##  0.05515955 0.18484045
## sample estimates:
## prop 1 prop 2 
##   0.57   0.45
\end{verbatim}

\hypertarget{sampling-distribution-of-the-sample-mean-and-standard-deviation-normal-and-chi2-distributions.}{%
\subsubsection{\texorpdfstring{4.2.6 Sampling distribution of the sample
mean and standard deviation; normal and \(\chi^2\)
distributions.}{4.2.6 Sampling distribution of the sample mean and standard deviation; normal and \textbackslash chi\^{}2 distributions.}}\label{sampling-distribution-of-the-sample-mean-and-standard-deviation-normal-and-chi2-distributions.}}

\begin{quote}
Suppose you draw n data points \(y_1, \dots, y_n\) from a normal
distribution with mean \(\mu\) and standard deviation \(\sigma\), and
then compute the sample mean
\(\bar y = \frac{1}{n} \sum_{i = 1}^n y_i\), and standard deviation
\(s_y = \sqrt{\frac{1}{n - 1} \sum_{i = 1}^n (y_i - \bar y)^2}\). These
two statistics have a sampling distribution that can be derived
mathematically from the properties of independent samples from the
normal. The sample mean, \(\bar y\), is normally distributed with mean
\(\mu\) and standard deviation \(\sigma / \sqrt n\). The sample standard
deviation has a distribution defined as follows:
\(s_y^2 \times (n - 1) / \sigma^2\) has a \(\chi^2\) distribution with
\(n - 1\) degrees of freedom.
\end{quote}

\hypertarget{degrees-of-freedom.}{%
\subsubsection{4.2.7 Degrees of freedom.}\label{degrees-of-freedom.}}

\begin{quote}
The concept of \emph{degrees of freedom} arises with the \(\chi^2\)
distribution and several other places in probability and statistics.
Without going into the technical details, we can briefly say that
degrees of freedom relate to the need to correct for overfitting when
estimating the error of future predictions from a fitted model\ldots{}
Roughly speaking, we can think of observed data as supplying \(n\)
``degrees of freedom'' that can be used for parameter estimation, and a
regression with \(k\) coefficients is said to use up \(k\) of these
degrees of freedom. (p.~53, \emph{emphasis} in the original)
\end{quote}

\hypertarget{confidence-intervals-from-the-t-distribution.}{%
\subsubsection{\texorpdfstring{4.2.8 Confidence intervals from the \(t\)
distribution.}{4.2.8 Confidence intervals from the t distribution.}}\label{confidence-intervals-from-the-t-distribution.}}

\begin{quote}
The \(t\) distribution is a family of symmetric distributions with
heavier tails (that is, a greater frequency of extreme values) compared
to the normal distribution. The \(t\) is characterized by a center, a
scale, and a degrees of freedom parameter that can range from 1 to
\(\infty\). Distributions in the \(t\) family with low degrees of
freedom have very heavy tails; in the other direction, in the limit as
degrees of freedom approach infinity, the \(t\) distribution approaches
the normal.

When a standard error is estimated from n data points, we can account
for uncertainty using the \(t\) distribution with \(n - 1\) degrees of
freedom, calcuated as \(n\) data points minus 1 because of the mean is
being estimated from the data. (p.~53)
\end{quote}

Take the case where \(y = \{35, 34, 38, 35, 37 \}\). Here are our \(n\),
mean, and standard deviation.

\begin{Shaded}
\begin{Highlighting}[]
\NormalTok{y }\OtherTok{\textless{}{-}} \FunctionTok{c}\NormalTok{(}\DecValTok{35}\NormalTok{, }\DecValTok{34}\NormalTok{, }\DecValTok{38}\NormalTok{, }\DecValTok{35}\NormalTok{, }\DecValTok{37}\NormalTok{)}

\NormalTok{(n }\OtherTok{\textless{}{-}} \FunctionTok{length}\NormalTok{(y))}
\end{Highlighting}
\end{Shaded}

\begin{verbatim}
## [1] 5
\end{verbatim}

\begin{Shaded}
\begin{Highlighting}[]
\NormalTok{(estimate }\OtherTok{\textless{}{-}} \FunctionTok{mean}\NormalTok{(y))}
\end{Highlighting}
\end{Shaded}

\begin{verbatim}
## [1] 35.8
\end{verbatim}

\begin{Shaded}
\begin{Highlighting}[]
\NormalTok{(s }\OtherTok{\textless{}{-}} \FunctionTok{sd}\NormalTok{(y))}
\end{Highlighting}
\end{Shaded}

\begin{verbatim}
## [1] 1.643168
\end{verbatim}

Now compute the standard error, along with the 50\% and 95\% intervals.

\begin{Shaded}
\begin{Highlighting}[]
\NormalTok{(se }\OtherTok{\textless{}{-}}\NormalTok{ s }\SpecialCharTok{/} \FunctionTok{sqrt}\NormalTok{(n))}
\end{Highlighting}
\end{Shaded}

\begin{verbatim}
## [1] 0.7348469
\end{verbatim}

\begin{Shaded}
\begin{Highlighting}[]
\CommentTok{\# 50\% CIs}
\NormalTok{estimate }\SpecialCharTok{+} \FunctionTok{qt}\NormalTok{(}\FunctionTok{c}\NormalTok{(}\FloatTok{0.25}\NormalTok{, }\FloatTok{0.75}\NormalTok{), }\AttributeTok{df =}\NormalTok{ n }\SpecialCharTok{{-}} \DecValTok{1}\NormalTok{) }\SpecialCharTok{*}\NormalTok{ se}
\end{Highlighting}
\end{Shaded}

\begin{verbatim}
## [1] 35.2557 36.3443
\end{verbatim}

\begin{Shaded}
\begin{Highlighting}[]
\CommentTok{\# 95\% CIs}
\NormalTok{estimate }\SpecialCharTok{+} \FunctionTok{qt}\NormalTok{(}\FunctionTok{c}\NormalTok{(}\FloatTok{0.025}\NormalTok{, }\FloatTok{0.975}\NormalTok{), }\AttributeTok{df =}\NormalTok{ n }\SpecialCharTok{{-}} \DecValTok{1}\NormalTok{) }\SpecialCharTok{*}\NormalTok{ se}
\end{Highlighting}
\end{Shaded}

\begin{verbatim}
## [1] 33.75974 37.84026
\end{verbatim}

\begin{Shaded}
\begin{Highlighting}[]
\CommentTok{\# use t.test}

\FunctionTok{t.test}\NormalTok{(y)}
\end{Highlighting}
\end{Shaded}

\begin{verbatim}
## 
##  One Sample t-test
## 
## data:  y
## t = 48.718, df = 4, p-value = 1.062e-06
## alternative hypothesis: true mean is not equal to 0
## 95 percent confidence interval:
##  33.75974 37.84026
## sample estimates:
## mean of x 
##      35.8
\end{verbatim}

\end{document}
